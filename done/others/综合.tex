\subsection{常用概念}

\subsection{欧拉路径}
	欧拉回路:每条边恰走一次的回路\par
	欧拉通路:每条边恰走一次的路径\par
	欧拉图:存在欧拉回路的图\par
	半欧拉图:存在欧拉通路的图\par
	有向欧拉图:每个点入度=出度\par
	无向欧拉图:每个点度数为偶数\par
	有向半欧拉图:一个点入度=出度+1,一个点入度=出度-1,其他点入度=出度\par
	无向半欧拉图:两个点度数为奇数,其他点度数为偶数\par

\subsection{映射}
[injective] or [one-to-one] 函数值不重复 \par {[}surjective] or [onto] 值域都被取到 \par {[}bijective] or [one-to-one correspondence] 一一对应

\subsection{反演}
反演中心$O$,反演半径$r$,点$p$的反演点$p'$满足$|OP||OP'|=r^2$\par
不经过反演中心的直线,反形为经过反演中心的圆\par
不经过反演中心的圆,反形为圆,反演中心为这两个互为反形的圆的位似中心\par

\subsection{弦图}
设 $next(v)$ 表示 $N(v)$ 中最前的点 . 
令 $w*$ 表示所有满足 $A \in B$ 的 $w$ 中最后的一个点 , 
判断 $v \cup N(v)$ 是否为极大团 , 
只需判断是否存在一个 $w \in w*$, 
满足 $Next(w)=v$ 且 $|N(v)| + 1 \leq |N(w)|$ 即可 . 

\subsection{五边形数}
$\prod_{n=1}^{\infty}{(1-x^{n})}=\sum_{n=0}^{\infty}{(-1)^{n}(1-x^{2n+1})x^{n(3n+1)/2}}$

\subsection{pick定理}
整多边形面积$A$=内部格点数$i$+边上格点数$\frac{b}{2}-1$\par

\subsection{重心}
半径为 $r$ , 圆心角为 $\theta$ 的扇形重心与圆心的距离为 $\frac{4r\sin(\theta/2)}{3\theta}$ \par
半径为 $r$ , 圆心角为 $\theta$ 的圆弧重心与圆心的距离为 $\frac{4r\sin^3(\theta/2)}{3(\theta-\sin(\theta))}$ \par

\subsection{第二类 Bernoulli number}
$B_m = 1 - \sum_{k=0}^{m-1}{\binom{m}{k}\frac{B_{k}}{m-k+1}}$\par
$S_m(n) = \sum_{k=1}^{n}{k^{m}} = \frac{1}{m+1}\sum_{k=0}^{m}{\binom{m+1}{k}B_{k}n^{m+1-k}}$\par

\subsection{Fibonacci 数}
$F_n=\frac{\varphi^{n}-(-\varphi)^{-n}}{\sqrt{5}},\varphi=\frac{1+\sqrt{5}}{2}$\par
$F_n=\lfloor \frac{\varphi^n}{\sqrt{5}}+\frac{1}{2}\rfloor$

\subsection{Catalan 数}
$C_{n+1}=\frac{2(2n+1)}{n+2}C_n$\par
$C_n=\frac{1}{n+1}\binom{2n}{n}=\frac{(2n)!}{(n+1)!n!}$\par
前20项:1, 1, 2, 5, 14, 42, 132, 429, 1430, 4862, 16796, 58786, 208012, 742900, 2674440, 9694845, 35357670, 129644790, 477638700, 1767263190\par
所有的奇卡塔兰数$C_n$都满足 $\displaystyle n=2^{k}-1$。所有其他的卡塔兰数都是偶数

\subsection{Lucas定理}
$C(n,m) mod p = C(n mod p, m mod p) * C(n / p, m / p),p$是质数\par

\subsection{扩展Lucas定理}
若$p$不是质数,将$p$分解质因数后分别求解,再用中国剩余定理合并\par

\subsection{BEST theorem}
有向图中欧拉回路的数量
 ${\displaystyle \operatorname {ec} (G)=t_{w}(G)\prod _{v\in V}{\bigl (}\deg(v)-1{\bigr )}!.} $ \par
其中$deg(v)$表示$v$的入度,$tw(G)$表示以$w$为根的外向树的数量,且在连通欧拉图中以任一点为根的外向树数量相同\par
若需要定起点,则答案乘上$deg(s)$,表示对每一条欧拉回路,$s$出现了$deg(s)$次,选取一个点切开得到一条从$s$出发的欧拉回路

\subsection{欧拉示性数定理}
对平面图$V-E+F=2$\par

\subsection{Polya定理}
设对$n$个对象用$m$种颜色: ${\displaystyle b_{1},b_{2},\cdots ,b_{m}}$着色。\par
设${\displaystyle m^{c(p_{i})}=(b_{1}+b_{2}+\cdots +b_{m})^{c_{1}(p_{i})}(b_{1}^{2}+b_{2}^{2}+\cdots +b_{m}^{2})^{c_{2}(p_{i})}\cdots (b_{1}^{n}+b_{2}^{n}+\cdots +b_{m}^{n})^{c_{n}(p_{i})}}$,其中${\displaystyle c_{j}(p_{i})}$表示置换群中第i个置换循环长度为j的个数。\par
设${\displaystyle S_{k}=(b_{1}^{k}+b_{2}^{k}+\cdots +b_{m}^{k}),k=1,2\cdots ,n}$,则波利亚计数定理的母函数形式为:${\displaystyle P(G)={\frac {1}{\mid G\mid }}\sum _{j=1}^{g}\Pi _{k=1}^{n}S_{k}^{c_{k}(p_{j})}}$\par
\subsection{Stirling 数}
第一类 :n 个元素的项目分作 k 个环排列的方法数目\par
$s(n, k) = (-1)^{n+k}|s(n, k)|$\par
$|s(n, 0)| =0$\par
$|s(1, 1)| =1$\par
$|s(n, k)| =|s(n-1, k-1)|+(n-1)*|s(n-1, k)|$\par
第二类 :n 个元素的集定义 k 个等价类的方法数\par
$    S(n,1)=S(n,n)=1$\par
 $   S(n,k)=S(n-1,k-1)+k*S(n-1,k)$\par

\subsection{常用排列组合公式}
$\sum_{i=1}^n x_i = k, x_i \geq 0$的解数为$C(n+k-1,n-1)$\par
$x_1 \geq 0,x_i \leq x_{i+1}, x_n \leq k-1$的解数等价于在[0,k-1]共k个数中可重复的取n个数的组合数,为$C(n+k-1,n)$\par

\subsection{三角公式}
$\sin(a \pm b) = \sin a \cos b \pm \cos a \sin b$\par
$\cos(a \pm b) = \cos a \cos b \mp \sin a \sin b$\par
$\tan(a \pm b) = \frac{\tan(a)\pm\tan(b)}{1 \mp \tan(a)\tan(b)}$\par
$\tan(a) \pm \tan(b) = \frac{\sin(a \pm b)}{\cos(a)\cos(b)}$\par
$\sin(a) + \sin(b) = 2\sin(\frac{a + b}{2})\cos(\frac{a - b}{2})$\par
$\sin(a) - \sin(b) = 2\cos(\frac{a + b}{2})\sin(\frac{a - b}{2})$\par
$\cos(a) + \cos(b) = 2\cos(\frac{a + b}{2})\cos(\frac{a - b}{2})$\par
$\cos(a) - \cos(b) = -2\sin(\frac{a + b}{2})\sin(\frac{a - b}{2})$\par
$\sin(na) = n\cos^{n-1}a\sin a - \binom{n}{3}\cos^{n-3}a \sin^3a + \binom{n}{5}\cos^{n-5}a\sin^5a - \dots$\par
$\cos(na) = \cos^{n}a - \binom{n}{2}\cos^{n-2}a \sin^2a + \binom{n}{4}\cos^{n-4}a\sin^4a - \dots$\par

\subsection{积分表}


== 含有$ax+b$的积分 ==\par

$\int (ax+b)^n\mbox{d}x=\frac{(ax+b)^{n+1}}{a(n+1)}+C$\par
$\int\frac{1}{ax+b}\mbox{d}x=\frac{1}{a}\ln \left | ax+b\right|+C$\par
$\int\frac{x}{ax+b}\mbox{d}x=\frac{1}{a^2}(ax+b-b\ln\left |ax+b \right|) +C$\par
$\int\frac{x^2}{ax+b}\mbox{d}x=\frac{1}{2a^3} \left[(ax+b)^2-4b(ax+b)+2b^2 \ln\left |ax+b\right| \right]+C$\par
$\int\frac{1}{x(ax+b)}\mbox{d}x = -\frac{1}{b}\ln\left | \frac{ax+b}{x}\right |+C$\par
$\int\frac{1}{x^2(ax+b)}\mbox{d}x=\frac{a}{b^2}\ln\left |\frac{ax+b}{x}\right |-\frac{1}{bx}+C$\par

==含有$\sqrt{a+bx}$的积分==\par
$\int x\sqrt{a+bx}\mbox{d}x=\frac{2}{15b^2}(3bx-2a)(a+bx)^{\frac{3}{2}}+C$\par
$\int x^2\sqrt{a+bx}\mbox{d}x=\frac{2}{105b^3}(15b^2x^2-12abx+8a^2)(a+bx)^{\frac{3}{2}}+C$\par
$\int x^n\sqrt{a+bx}\mbox{d}x=\frac{2}{b(2n+3)}x^n(a+bx)^{\frac{3}{2}} -\frac{2na}{b(2n+3)}\int x^{n-1}\sqrt{a+bx}\mbox{d}x$\par
$\int\frac{\sqrt{a+bx}}{x}\mbox{d}x=2\sqrt{a+bx}+a\int\frac{1}{x\sqrt{a+bx}}\mbox{d}x$\par
$\int\frac{\sqrt{a+bx}}{x^n}\mbox{d}x=\frac{-1}{a(n-1)}\frac{(a+bx)^{\frac{3}{2}}}{x^{n-1}} -\frac{(2n-5)b}{2a(n-1)}\int\frac{\sqrt{a+bx}}{x^{n-1}}\mbox{d}x,n\neq 1$\par
$\int\frac{1}{x\sqrt{a+bx}}\mbox{d}x=\frac{1}{\sqrt{a}}\ln\left (\frac{\sqrt{a+bx} -\sqrt{a}}{\sqrt{a+bx}+\sqrt{a}}\right )+C,a>0
=\frac{2}{\sqrt{-a}}\arctan\sqrt\frac{a+bx}{-a} +C,a<0$\par
$\int\frac{1}{x^n\sqrt{a+bx}}\mbox{d}x=\frac{-1}{a(n-1)}\frac{\sqrt{a+bx}}{x^{n-1}} -\frac{(2n-3)b}{2a(n-1)}\int\frac{1}{x^{n-1}}\sqrt{a+bx}\mbox{d}x,n\neq 1$\par
== 含有$x^2\pm\alpha^2$的积分 ==\par
$\int\frac{1}{x^2+\alpha^2}\mbox{d}x=\frac{\arctan\dfrac{x}{\alpha}}{\alpha}+C$\par
$\int\frac{1}{\pm x^2\mp\alpha^2}\mbox{d}x = \frac{\ln\left(\dfrac{x\mp\alpha}{\pm x+\alpha}\right)}{2\alpha}+C$\par
== 含有 ${ax^2+b}$的积分 ==\par
$\int\frac{1}{ax^2+b}\mbox{d}x=\frac{1}{\sqrt{ab}} \arctan\frac{\sqrt{a}x}{\sqrt{b}}+C$\par
== 含有 $ax^2+bx+c\qquad(a>0)$的积分 ==\par
$\int ax^2+bx+c\mbox{d}x=\frac{ax^3}{3}+\frac{bx^2}{2}+cx+C$\par
==含有 $\sqrt{a^2+x^2}\qquad(a>0)$的积分==\par
$\int\sqrt{a^2+x^2}\mbox{d}x=\frac{1}{2}x\sqrt{a^2+x^2}+\frac{1}{2}a^2\ln\left (x+\sqrt{a^2+x^2}\right )+C$\par
$\int x^2\sqrt{a^2+x^2}\mbox{d}x=\frac{1}{8}x(a^2+2x^2)\sqrt{a^2+x^2}-\frac{1}{8}a^4\ln\left (x+\sqrt{a^2+x^2}\right )+C$\par
$\int \frac{\sqrt{a^2+x^2}}{x}\mbox{d}x = \sqrt{a^2+x^2} - a\ln \left ( \frac{a+\sqrt{a^2+x^2}}{x} \right ) +C$\par
$\int \frac{\sqrt{a^2+x^2}}{x^2}\mbox{d}x = \ln\left ( x+\sqrt{a^2+x^2}\right ) -  \frac{\sqrt{a^2+x^2}}{x} +C$\par
$\int \frac{1}{\sqrt{a^2+x^2}}\mbox{d}x=\ln \left ( x+\sqrt{a^2+x^2} \right ) +C$\par
$\int\frac{x^2}{\sqrt{a^2+x^2}}\mbox{d}x=\frac{1}{2}x\sqrt{a^2+x^2} -\frac{1}{2}a^2\ln\left (\sqrt{a^2+x^2}+x \right )+C$\par
$\int\frac{1}{x\sqrt{a^2+x^2}}\mbox{d}x=\frac{1}{a}\ln\left (\frac{x}{a+\sqrt{a^2+x^2}}\right )+C$\par
$\int\frac{1}{x^2\sqrt{a^2+x^2}}\mbox{d}x=-\frac{\sqrt{a^2+x^2}}{a^2x}+C$\par

==含有$\sqrt{x^2-a^2}\qquad{(x^2>a^2)}$的积分=\par
$\int \frac{1}{\sqrt{x^2-a^2}}\mbox{d}x=\ln| x+\sqrt{x^2-a^2} | +C$\par

==含有$\sqrt{a^2-x^2}\qquad(a^2>x^2)$的积分==\par
$\int \frac{1}{\sqrt{a^2-x^2}}\mbox{d}x= \arcsin \frac{x}{a} +C = - \arccos \frac{x}{a} +C$\par
$\int\sqrt{a^2-x^2}\mbox{d}x=\frac{1}{2}x\sqrt{a^2-x^2} +\frac{a^2}{2}\arcsin\frac{x}{a}+C$\par
$\int x^2\sqrt{a^2-x^2}\mbox{d}x=\frac{1}{8}x(2x^2-a^2)\sqrt{a^2-x^2} +\frac{1}{8}a^4\arcsin\frac{x}{a}+C$\par
$\int\frac{\sqrt{a^2-x^2}}{x}\mbox{d}x=\sqrt{a^2-x^2} -a\ln\left (\frac{a+\sqrt{a^2-x^2}}{x}\right )+C$\par
$\int\frac{\sqrt{a^2-x^2}}{x^2}\mbox{d}x=-\frac{\sqrt{a^2-x^2}}{x} -\arcsin\frac{x}{a}+C$\par
$\int\frac{1}{x\sqrt{a^2-x^2}}\mbox{d}x=-\frac{1}{a}\ln\left (\frac{a+\sqrt{a^2-x^2}}{x}\right )+C$\par
$\int\frac{x^2}{\sqrt{a^2-x^2}}\mbox{d}x=-\frac{1}{2}x\sqrt{a^2-x^2}+\frac{1}{2}a^2\arcsin\frac{x}{a}+C$\par
$\int\frac{1}{x^2\sqrt{a^2-x^2}}\mbox{d}x=-\frac{\sqrt{a^2-x^2}}{a^2x}+C$\par
==含有$R=\sqrt{|a|x^2+bx+c}\qquad(a\ne0)$的积分==\par
$\int\frac{\mbox{d}x}{R} = \frac{1}{\sqrt{a}}\ln\left(2\sqrt{a}R+2ax+b\right)\qquad(\mbox{for }a>0)$\par
$\int\frac{\mbox{d}x}{R} = \frac{1}{\sqrt{a}}\,\operatorname{arsinh}\frac{2ax+b}{\sqrt{4ac-b^2}} \qquad \mbox{(for }a>0\mbox{, }4ac-b^2>0\mbox{)}$\par
$\int\frac{\mbox{d}x}{R} = \frac{1}{\sqrt{a}}\ln|2ax+b| \quad \mbox{(for }a>0\mbox{, }4ac-b^2=0\mbox{)}$\par
$\int\frac{\mbox{d}x}{R} = -\frac{1}{\sqrt{-a}}\arcsin\frac{2ax+b}{\sqrt{b^2-4ac}} \qquad \mbox{(for }a<0\mbox{, }4ac-b^2<0\mbox{, }\left(2ax+b\right)<\sqrt{b^2-4ac}\mbox{)}$\par
$\int\frac{\mbox{d}x}{R^3} = \frac{4ax+2b}{(4ac-b^2)R}$\par
$\int\frac{\mbox{d}x}{R^5} = \frac{4ax+2b}{3(4ac-b^2)R}\left(\frac{1}{R^2}+\frac{8a}{4ac-b^2}\right)$\par
$\int\frac{\mbox{d}x}{R^{2n+1}} = \frac{2}{(2n-1)(4ac-b^2)}\left[\frac{2ax+b}{R^{2n-1}}+4a(n-1)\int\frac{\mbox{d}x}{R^{2n-1}}\right]$\par
$\int\frac{x}{R}\;\mbox{d}x = \frac{R}{a}-\frac{b}{2a}\int\frac{\mbox{d}x}{R}$\par
$\int\frac{x}{R^3}\;\mbox{d}x = -\frac{2bx+4c}{(4ac-b^2)R}$\par
$\int\frac{x}{R^{2n+1}}\;\mbox{d}x = -\frac{1}{(2n-1)aR^{2n-1}}-\frac{b}{2a}\int\frac{\mbox{d}x}{R^{2n+1}}$\par
$\int\frac{\mbox{d}x}{xR}=-\frac{1}{\sqrt{c}}\ln\left(\frac{2\sqrt{c}R+bx+2c}{x}\right)$\par
$\int\frac{\mbox{d}x}{xR}=-\frac{1}{\sqrt{c}}\operatorname{arsinh}\left(\frac{bx+2c}{|x|\sqrt{4ac-b^2}}\right)$\par
== 含有三角函数的积分 ==\par
$\int\cos x\mbox{d}x=\sin x+C$\par
$\int\sin x\mbox{d}x= -\cos x+C$\par
$\int\sec^2x\mbox{d}x=\tan x+C$\par
$\int\csc^2x\mbox{d}x=-\cot x+C$\par
$\int\sec x\tan x\mbox{d}x=\sec x+C$\par
$\int\csc x\cot x\mbox{d}x=-\csc x+C$\par
$\int\tan x\mbox{d}x=-\ln{\left|\cos {x}\right|}+C=\ln{\left|\sec x\right|}+C$\par
$\int\cot x\mbox{d}x=\ln{\left|\sin x\right|}+C$\par
$\int\sec x\mbox{d}x=\ln{\left|\sec x+\tan x\right|}+C$\par
$\int\csc x\mbox{d}x=-\ln{\left|\csc x+\cot x\right|}+C=\ln{\left|{\tan x-\sin x\over\sin x\tan x}\right|}+C$\par
$\int \sin ^n x \mbox{d}x = - \frac{1}{n} \sin ^{n-1} x \cos x + \frac{n-1}{n} \int \sin ^{n-2} x \mbox{d}x +C \quad \forall n \ge 2$\par
$\int \sin ^2 x \mbox{d}x = \frac{x}{2}-\frac{\sin{2x}}{4} +C$\par
$\int \cos ^n x \mbox{d}x = \frac{1}{n} \cos ^{n-1} x \sin x + \frac{n-1}{n} \int \cos ^{n-2} x \mbox{d}x +C \quad \forall n \ge 2$\par
$\int \cos ^2 x \mbox{d}x = \frac{x}{2}+\frac{\sin{2x}}{4} +C$\par
$\int \tan ^n x \mbox{d}x = \frac{1}{n-1} \tan ^{n-1} x - \int \tan ^{n-2} x \mbox{d}x +C \quad \forall n \ge 2$\par
$\int \tan ^2 x \mbox{d}x = \tan x - x +C$\par
$\int \cot ^n x \mbox{d}x = \frac{1}{n-1} \cot ^{n-1} x - \int \cot ^{n-2} x \mbox{d}x +C \quad \forall n \ge 2$\par
$\int \cot ^2 x \mbox{d}x = - \cot x - x +C$\par
$\int \sec ^n x \mbox{d}x = \frac{1}{n-1} \sec ^{n-2} x \tan x + \frac{n-2}{n-1} \int \sec ^{n-2} x \mbox{d}x +C \quad \forall n \ge 2$\par
$\int \csc ^n x \mbox{d}x = - \frac{1}{n-1} \csc ^{n-2} x \cot x + \frac{n-2}{n-1} \int \csc ^{n-2} x \mbox{d}x +C \quad \forall n \ge 2$\par
== 含有反三角函数的积分 ==\par
$\int \arcsin x \mbox{d}x = x \arcsin x + \sqrt {1 - x^2} +C$\par
$\int \arccos x \mbox{d}x = x \arccos x - \sqrt {1 - x^2} +C$\par
$\int \arctan x \mbox{d}x = x \arctan x - \ln \sqrt {1 + x^2} +C$\par
$\int arccot(x) \mbox{d}x = x\times  arccot (x) + \ln \sqrt {1 + x^2} +C$\par
$\int arcsec(x) \mbox{d}x = x\times arcsec (x) - sgn(x)\ln \left|x + \sqrt{x^2 - 1}\right| +C
                              = x\times  arcsec(x) + sgn(x)\ln \left|x - \sqrt{x^2 - 1}\right| +C$\par
$\int arccsc (x) \mbox{d}x = x\times  arccsc(x) + sgn(x)\ln \left|x + \sqrt{x^2 - 1}\right| +C
                               = x\times  arccsc (x) - sgn(x)\ln \left|x - \sqrt{x^2 - 1}\right| +C$\par
== 含有指数函数的积分 ==\par
$\int e^x\mbox{d}x=e^x+C$\par
$\int\alpha^x\mbox{d}x=\frac{\alpha^x}{\ln\alpha}+C$\par
$\int xe^{ax}\mbox{d}x=\frac{1}{a^2}(ax-1)e^{ax}+C$\par
$\int x^ne^{ax}\mbox{d}x=\frac{1}{a}x^ne^{ax}-\frac{n}{a}\int x^{n-1}e^{ax}\mbox{d}x$\par
$\int e^{ax}\sin bx \mbox{d}x=\frac{e^{ax}}{a^2+b^2}(a\sin bx-b\cos bx)+C$\par
$\int e^{ax}\cos bx \mbox{d}x=\frac{e^{ax}}{a^2+b^2}(a\cos bx+b\sin bx)+C$\par
== 含有对数函数的积分 ==\par
$\int\ln x\mbox{d}x = x\ln x - x + C$\par
$\int\log_\alpha x\mbox{d}x=\frac{1}{\ln\alpha}\left({x\ln x - x}\right)+C$\par
$\int x^n\ln x\mbox{d}x = \frac{x^{n+1}}{(n+1)^2}[(n+1)\ln x -1]+ C$\par
$\int\frac{1}{x\ln{x}}\mbox{d}x = \ln{(\ln{x})}+C$\par
== 含有双曲函数的积分 ==\par
$\int \sinh x \mbox{d}x = \cosh x +C$\par
$\int \cosh x \mbox{d}x = \sinh x +C$\par
$\int \tanh x \mbox{d}x = \ln\left(\cosh x\right) +C$\par
$\int \coth x \mbox{d}x = \ln\left(\sinh x\right) +C$\par
$\int \mbox{sech}\ x \mbox{d}x = \arcsin\left(\tanh x\right) + C =  \arctan\left(\sinh x\right) + C $\par
$\int \mbox{csch}\ x \mbox{d}x = \ln\left(\tanh {x \over2}\right) + C$\par
== 定积分 ==\par
$\int^\infty_{-\infty}e^{-\alpha x^2}\mbox{d}x=\sqrt{\frac{\pi}{\alpha}}$\par
$\int_0^\frac{\pi}{2} \mbox{sin}^n x\mbox{d}x=\int_0^\frac{\pi}{2} \mbox{cos}^n x\mbox{d}x=$\par
$\begin{cases}$
$\frac{n-1}{n}\cdot\frac{n-3}{n-2}\cdot\ldots\cdot\frac{4}{5}\cdot\frac{2}{3}, & \mbox{if }n>1\mbox{ 且 }n\mbox{ 为 奇 数} \\
\frac{n-1}{n}\cdot\frac{n-3}{n-2}\cdot\ldots\cdot\frac{3}{4}\cdot\frac{1}{2}\cdot\frac{\pi}{2}, & \mbox{if }n>0\mbox{ 且 }n\mbox{ 为 偶 数}$
$\end{cases}$


