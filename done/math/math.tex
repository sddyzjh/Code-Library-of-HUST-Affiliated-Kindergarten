
\subsection{常见积性函数}

单位函数$e(x)= \begin{cases}
	1,x=1 \\ 0,x>1
\end{cases}$\par

常函数$I(x)=1$\par

幂函数$id(x)=x^k$\par

欧拉函数$\varphi(x)=x\prod_{p|x}(1-\frac{1}{p})$\par

$n\geq 2$时$\varphi(n)$为偶数\par

莫比乌斯函数$\mu(x)=\begin{cases}
	1,x=1\\(-1)^k,x=p_1p_2...p_k \\ 0,others
\end{cases}$\par

$\sigma_k(n) = \Sigma_{d|n}d^k$\par

$\sigma_k(n)=\Pi_{i=1}^{num}\frac{(p_i^{a_i+1})^k-1}{p_i^k-1}$\par

\subsection{常用公式}

$\Sigma_{d|n} \varphi (n)=n\rightarrow \varphi(n)=n-\Sigma_{d|n,d<n}$\par

$[n=1]=\Sigma_{d|n}\mu(d)$排列组合后二项式定理转换即可证明\par

$n=\Sigma_{d|n}\varphi(d)$将$\frac{i}{n}(1\leq i\leq n)$化为最简分数统计个数即可证明\par

\subsection{狄利克雷卷积}

$h(n)=\sum_{d|n}f(d)g(\frac{n}{d})$称为f和g的狄利克雷卷积,也可以理解为$h(n)=\sum_{ij=n}f(i)g(j)$\par

两个积性函数的狄利克雷卷积仍为积性函数\par

狄利克雷卷积满足交换律和结合律\par

\subsection{莫比乌斯反演}

$f(n)=\sum_{d|n}g(d)\Rightarrow g(n)=\sum_{d|n}\mu(d)*f(\frac{n}{d})$\par
即$f=g*I \Leftrightarrow g=\mu*f$\par
$\mu*I=e$\par
$f=g*I \Rightarrow \mu*f = g*(\mu*I)=g*e=g$\par
$g=\mu*f \Rightarrow f=g*I$\par

$F(n)=\sum_{n|d}f(d)\Rightarrow f(n)=\sum_{n|d}\mu(\frac{n}{d})*F(d)$\par

$f(n)=\sum_{d|n}\phi(d)\Rightarrow \phi(n)=\sum_{d|n}\mu(d)f(\frac{n}{d})=\sum_{d|n}\mu(d)\frac{n}{d}$\par

\subsection{常用等式}

$\varphi = \mu * id$\par

$\varphi * I = id$\par

$\sum_{d|N}\phi(d)=N$\par

$\sum_{i\leq N}i*[(i,N)=1]=\frac{N*\phi(N)}{2}$\par

$\sum_{d|N}\frac{\mu(d)}{d}=\frac{\phi(N)}{N}$\par

常用代换\par

$\sum_{d|N}\mu(d)=[N=1]$\par

考虑每个数的贡献\par

$\sum_{i\leq N}\lfloor \frac{N}{i}\rfloor=\sum_{i\leq N}d(i)$\par

\subsection{Pell方程}

形如$x^2-dy^2=1$的方程\par

当d为完全平方数时无解\par

假设$(x_0,y_0)$为最小正整数解\par

$x_n = x_{n-1} \times x_0 + d \times y_{n-1} \times y_0$\par

$y_n = x_{n-1} \times y_0 + y_{n-1} \times x_0$
