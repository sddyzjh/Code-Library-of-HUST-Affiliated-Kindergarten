\documentclass[a4paper]{article}
%\setlength\paperheight {297mm}%
%\setlength\paperwidth  {250mm}%
\usepackage{listings}
%\usepackage{CJK}
\usepackage[UTF8]{ctex}
\usepackage{xcolor}
\usepackage{amsmath}
\setCJKmainfont{微软雅黑}
\begin{document}
\lstset{language=C++}          % Set your language (you can change the language for each code-block optionally)
%\begin{CJK*}{UTF8}{song}
\lstset{
	basicstyle=\footnotesize,  				%字号
	breaklines=true,						%自动换行
	frame=single,						%背景框
%	linewidth=14cm,
%	xleftmargin=0.6\textwidth, xrightmargin=0.6\textwidth,
	numbers=left,							%行号
	tabsize=4,	
	commentstyle=\color[rgb]{0.133,0.545,0.133},	%注释
	%identifierstyle=\color{red},				%标识符
	keywordstyle=\color[rgb]{0,0,1},			%关键字
	numberstyle=\color[RGB]{0,192,192},			%数字
	stringstyle=\color[rgb]{0.627,0.126,0.941},	%字符串
%	literate=
        %   *{\{}{{\textcolor{red}{\{}}}{1}
       %     {\}}{{\textcolor{red}{\}}}}{1}
      %      {[}{{\textcolor{red}{[}}}{1}
     %       {]}{{\textcolor{red}{]}}}{1}
    %        {(}{{\textcolor{red}{(}}}{1}
   %         {)}{{\textcolor{red}{)}}}{1}
  %          {;}{{\textcolor{red}{;}}}{1}
 %           {,}{{\textcolor{red}{,}}}{1}
%            {.}{{\textcolor{red}{.}}}{1}
%            {+}{{\textcolor{red}{+}}}{1}
%            {-}{{\textcolor{red}{-}}}{1}
%            {*}{{\textcolor{red}{*}}}{1}
%            {/}{{\textcolor{red}{/}}}{1}
%            {=}{{\textcolor{red}{=}}}{1}
%            {<}{{\textcolor{red}{<}}}{1}
%            {>}{{\textcolor{red}{>}}}{1}
%            {~}{{\textcolor{red}{~}}}{1}
%            {!}{{\textcolor{red}{!}}}{1}
%     		 {@}{{\textcolor{red}{@}}}{1}
%            {\#}{{\textcolor{red}{\#}}}{1}
%            {\$}{{\textcolor{red}{\$}}}{1}
%            {\%}{{\textcolor{red}{\%}}}{1}
%            {\^}{{\textcolor{red}{\^}}}{1}
%            {\&}{{\textcolor{red}{\&}}}{1}
}
\newpage
\begin{titlepage} % Suppresses headers and footers on the title page

	\centering % Centre everything on the title page
	
	\scshape % Use small caps for all text on the title page
	
	\vspace*{\baselineskip} % White space at the top of the page
	
	%------------------------------------------------
	%	Title
	%------------------------------------------------
	
	\rule{\textwidth}{1.6pt}\vspace*{-\baselineskip}\vspace*{2pt} % Thick horizontal rule
	\rule{\textwidth}{0.4pt} % Thin horizontal rule
	
	\vspace{0.75\baselineskip} % Whitespace above the title
	
	{\LARGE Standard Code Library\\ of\\ HUST Affiliated Kindergarten\\} % Title
	
	\vspace{0.75\baselineskip} % Whitespace below the title
	
	\rule{\textwidth}{0.4pt}\vspace*{-\baselineskip}\vspace{3.2pt} % Thin horizontal rule
	\rule{\textwidth}{1.6pt} % Thick horizontal rule
	
	\vspace{2\baselineskip} % Whitespace after the title block
	
	%------------------------------------------------
	%	Subtitle
	%------------------------------------------------
	
	Verson 1.2 % Subtitle or further description
	
	\vspace*{3\baselineskip} % Whitespace under the subtitle
	
	%------------------------------------------------
	%	Editor(s)
	%------------------------------------------------
	
	Edited By
	
	\vspace{0.5\baselineskip} % Whitespace before the editors
	
	{\protect sddyzjh \\ DragoonKiller \\ dreactor \\} % Editor list
	
	\vspace{0.5\baselineskip} % Whitespace below the editor list
	
	\textit{Huazhong University of Science and Technology} % Editor affiliation
	
	\vfill % Whitespace between editor names and publisher logo
	
	%------------------------------------------------
	%	Publisher
	%------------------------------------------------
	
\date{\today}
%	2017.09.01 % Publication year
	

\end{titlepage}
%remove leading numbers in table of contents
\setcounter{secnumdepth}{0}

\tableofcontents
\newpage
\section{计算几何}
\subsection{几何通用}
\lstinputlisting[language=C++]{calculategeometry/几何通用.cpp}
\subsection{平面几何通用}
\lstinputlisting[language=C++]{calculategeometry/平面几何通用.cpp}
\subsection{立体几何通用}
\lstinputlisting[language=C++]{calculategeometry/立体几何通用.cpp}
\subsection{判断点在凸多边形内}
\lstinputlisting[language=C++]{calculategeometry/判断点在凸多边形内.cpp}
\subsection{凸包}
\lstinputlisting[language=C++]{calculategeometry/凸包.cpp}
\subsection{旋转卡壳}
\lstinputlisting[language=C++]{calculategeometry/旋转卡壳.cpp}
\subsection{最小覆盖圆}
\lstinputlisting[language=C++]{calculategeometry/最小覆盖圆.cpp}
\section{数据结构}
\subsection{KD树}
\lstinputlisting[language=C++]{datastructure/KD树.cpp}
\subsection{Splay}
\lstinputlisting[language=C++]{datastructure/Splay.cpp}
\subsection{表达式解析}
\lstinputlisting[language=C++]{datastructure/表达式解析.cpp}
\subsection{并查}
\lstinputlisting[language=C++]{datastructure/并查.cpp}
\subsection{可持久化线段树}
\lstinputlisting[language=C++]{datastructure/可持久化线段树.cpp}
\subsection{轻重边剖分}
\lstinputlisting[language=C++]{datastructure/轻重边剖分.cpp}
\subsection{手写bitset}
\lstinputlisting[language=C++]{datastructure/手写bitset.cpp}
\subsection{树状数组}
\lstinputlisting[language=C++]{datastructure/树状数组.cpp}
\subsection{线段树}
\lstinputlisting[language=C++]{datastructure/线段树.cpp}
\subsection{左偏树}
\lstinputlisting[language=C++]{datastructure/左偏树.cpp}
\section{动态规划}
\subsection{插头DP}
\lstinputlisting[language=C++]{dp/插头dp.cpp}
\subsection{概率DP}
\lstinputlisting[language=C++]{dp/概率dp.cpp}
\subsection{数位DP}
\lstinputlisting[language=C++]{dp/数位dp.cpp}
\subsection{四边形DP}
\lstinputlisting[language=C++]{dp/四边形dp.cpp}
\subsection{斜率DP}
\lstinputlisting[language=C++]{dp/斜率dp.cpp}
\subsection{状压DP}
\lstinputlisting[language=C++]{dp/状压dp.cpp}
\subsection{最长上升子序列}
\lstinputlisting[language=C++]{dp/最长上升子序列.cpp}
\section{图论}
\subsection{best's therom}
\lstinputlisting[language=C++]{graphtheory/best_therom.cpp}
\subsection{k短路可持久化堆}
\lstinputlisting[language=C++]{graphtheory/k短路可持久化堆.cpp}
\subsection{spfa费用流}
\lstinputlisting[language=C++]{graphtheory/spfa费用流.cpp}
\subsection{Tarjan有向图强连通分量}
\lstinputlisting[language=C++]{graphtheory/Tarjan有向图强连通分量.cpp}
\subsection{zkw费用流}
\lstinputlisting[language=C++]{graphtheory/zkw费用流.cpp}
\subsection{倍增LCA}
\lstinputlisting[language=C++]{graphtheory/倍增LCA.cpp}
\subsection{点分治}
\lstinputlisting[language=C++]{graphtheory/点分治.cpp}
\subsection{堆优化dijkstra}
\lstinputlisting[language=C++]{graphtheory/堆优化dijkstra.cpp}
\subsection{矩阵树定理}
\lstinputlisting[language=C++]{graphtheory/矩阵树定理.cpp}
\subsection{平面欧几里得距离最小生成树}
\lstinputlisting[language=C++]{graphtheory/平面欧几里得距离最小生成树.cpp}
\subsection{最大流Dinic}
\lstinputlisting[language=C++]{graphtheory/最大流Dinic.cpp}
\subsection{最大团}
\lstinputlisting[language=C++]{graphtheory/最大团.cpp}
\subsection{最小度限制生成树}
\lstinputlisting[language=C++]{graphtheory/最小度限制生成树.cpp}
\subsection{最优比率生成树}
\lstinputlisting[language=C++]{graphtheory/最优比率生成树.cpp}
\section{数学}
\subsection{常用公式}
\subsection{积性函数}
$\sigma_k(n) = \Sigma_{d|n}d^k$表示$n$的约数的$k$次幂和\par
$\sigma_k(n)=\Pi_{i=1}^{k}\frac{(p_i^{a_i+1})^k-1}{p_i^k-1}$\par
$\varphi (n) = \Sigma_{i=1}^{n}[(n,i)=1]=\Pi_{i=1}^{k}(1-\frac{1}{p_i})$\par
$\varphi (p^k) = (p-1)p^{k-1}$\par
$\Sigma_{d|n} \varphi (n)=n\rightarrow \varphi(n)=n-\Sigma_{d|n,d<n}$\par
$n\geq 2$时$\varphi(n)$为偶数\par 
$\mu(n)=\left\{
                \begin{array}{ll}
                  0 \quad \quad $有平方因子$\\
                  (-1)^t \quad n=\Pi_{i=1}^{t}p_i
                \end{array}
              \right.
$\par
$[n=1]=\Sigma_{d|n}\mu(d)$排列组合后二项式定理转换即可证明\par
$n=\Sigma_{d|n}\varphi(d)$将$\frac{i}{n}(1\leq i\leq n)$化为最简分数统计个数即可证明\par
\subsection{莫比乌斯反演}
$F(n)=\sum_{d|n}f(d)\Rightarrow f(n)=\sum_{d|n}\mu(d)*F(\frac{n}{d})$\par
$F(n)=\sum_{n|d}f(d)\Rightarrow f(n)=\sum_{n|d}\mu(\frac{n}{d})*F(d)$\par
$f(n)=\sum_{d|n}\phi(d)\Rightarrow \phi(n)=\sum_{d|n}\mu(d)f(\frac{n}{d})=\sum_{d|n}\mu(d)\frac{n}{d}$\par
\subsection{常用等式}
不知道有什么用\par
$\sum_{d|N}\phi(d)=N$\par
$\sum_{i\leq N}i*[(i,N)=1]=\frac{N*\phi(N)}{2}$\par
$\sum_{d|N}\frac{\mu(d)}{d}=\frac{\phi(N)}{N}$\par
常用代换\par
$\sum_{d|N}\mu(d)=[N=1]$\par
考虑每个数的贡献\par
$\sum_{i\leq N}\lfloor \frac{N}{i}\rfloor=\sum_{i\leq N}d(i)$\par

\subsection{SG函数}
\lstinputlisting[language=C++]{math/SG函数.cpp}
\subsection{矩阵乘法快速幂}
\lstinputlisting[language=C++]{math/矩阵乘法快速幂.cpp}
\subsection{线性基}
\lstinputlisting[language=C++]{math/线性基.cpp}
\subsection{线性筛}
\lstinputlisting[language=C++]{math/线性筛.cpp}
\subsection{整数卷积NTT}
\lstinputlisting[language=C++]{math/整数卷积NTT.cpp}
\subsection{中国剩余定理}
\lstinputlisting[language=C++]{math/中国剩余定理.cpp}
\section{字符串}
\subsection{AC自动机}
\lstinputlisting[language=C++]{string/AC自动机.cpp}
\subsection{KMP}
\lstinputlisting[language=C++]{string/KMP.cpp}
\subsection{Manacher}
\lstinputlisting[language=C++]{string/Manacher回文串.cpp}
\subsection{Trie树}
\lstinputlisting[language=C++]{string/Trie树.cpp}
\subsection{后缀数组-DC3}
\lstinputlisting[language=C++]{string/后缀数组-DC3.cpp}
\subsection{后缀数组-倍增法}
\lstinputlisting[language=C++]{string/后缀数组-倍增法.cpp}
\subsection{后缀自动机}
\lstinputlisting[language=C++]{string/后缀自动机.cpp}
\subsection{扩展KMP}
\lstinputlisting[language=C++]{string/扩展KMP.cpp}
\section{杂项}
\subsection{测速}
\lstinputlisting[language=C++]{others/chrono.cpp}
\subsection{日期公式}
\lstinputlisting[language=C++]{others/date.cpp}
\subsection{读入挂}
\lstinputlisting[language=C++]{others/fread.cpp}
\subsection{高精度}
\lstinputlisting[language=C++]{others/高精度.cpp}
\subsection{康托展开与逆展开}
\lstinputlisting[language=C++]{others/康托展开与康托逆展开.cpp}
\subsection{快速乘}
\lstinputlisting[language=C++]{others/快速乘.cpp}
\subsection{模拟退火}
\lstinputlisting[language=C++]{others/模拟退火.cpp}
\subsection{常用概念}
\subsection{映射}
[injective] or [one-to-one] 函数值不重复 \par {[}surjective] or [onto] 值域都被取到 \par {[}bijective] or [one-to-one correspondence] 一一对应
\subsection{反演}
反演中心$O$,反演半径$r$,点$p$的反演点$p'$满足$|OP||OP'|=r^2$\par
不经过反演中心的直线,反形为经过反演中心的圆\par
不经过反演中心的圆,反形为圆,反演中心为这两个互为反形的圆的位似中心\par
\subsection{弦图}
设 $next(v)$ 表示 $N(v)$ 中最前的点 . 
令 $w*$ 表示所有满足 $A \in B$ 的 $w$ 中最后的一个点 , 
判断 $v \cup N(v)$ 是否为极大团 , 
只需判断是否存在一个 $w \in w*$, 
满足 $Next(w)=v$ 且 $|N(v)| + 1 \leq |N(w)|$ 即可 . 
\subsection{五边形数}
$\prod_{n=1}^{\infty}{(1-x^{n})}=\sum_{n=0}^{\infty}{(-1)^{n}(1-x^{2n+1})x^{n(3n+1)/2}}$
\subsection{pick定理}
整多边形面积$A$=内部格点数$i$+边上格点数$\frac{b}{2}-1$\par
\subsection{重心}
半径为 $r$ , 圆心角为 $\theta$ 的扇形重心与圆心的距离为 $\frac{4r\sin(\theta/2)}{3\theta}$ \par
半径为 $r$ , 圆心角为 $\theta$ 的圆弧重心与圆心的距离为 $\frac{4r\sin^3(\theta/2)}{3(\theta-\sin(\theta))}$ \par
\subsection{第二类 Bernoulli number}
$B_m = 1 - \sum_{k=0}^{m-1}{\binom{m}{k}\frac{B_{k}}{m-k+1}}$\par
$S_m(n) = \sum_{k=1}^{n}{k^{m}} = \frac{1}{m+1}\sum_{k=0}^{m}{\binom{m+1}{k}B_{k}n^{m+1-k}}$\par
\subsection{Fibonacci 数}
$F_n=\frac{\varphi^{n}-(-\varphi)^{-n}}{\sqrt{5}},\varphi=\frac{1+\sqrt{5}}{2}$\par
$F_n=\lfloor \frac{\varphi^n}{\sqrt{5}}+\frac{1}{2}\rfloor$
\subsection{Catalan 数}
$C_{n+1}=\frac{2(2n+1)}{n+2}C_n$\par
$C_n=\frac{1}{n+1}\binom{2n}{n}=\frac{(2n)!}{(n+1)!n!}$\par
前20项:1, 1, 2, 5, 14, 42, 132, 429, 1430, 4862, 16796, 58786, 208012, 742900, 2674440, 9694845, 35357670, 129644790, 477638700, 1767263190\par
所有的奇卡塔兰数$C_n$都满足 $\displaystyle n=2^{k}-1$。所有其他的卡塔兰数都是偶数
\subsection{Stirling 数}
第一类 :n 个元素的项目分作 k 个环排列的方法数目\par
$s(n, k) = (-1)^{n+k}|s(n, k)|$\par
$|s(n, 0)| =0$\par
$|s(1, 1)| =1$\par
$|s(n, k)| =|s(n-1, k-1)|+(n-1)*|s(n-1, k)|$\par
第二类 :n 个元素的集定义 k 个等价类的方法数\par
$    S(n,1)=S(n,n)=1$\par
 $   S(n,k)=S(n-1,k-1)+k*S(n-1,k)$\par
\subsection{三角公式}
$\sin(a \pm b) = \sin a \cos b \pm \cos a \sin b$\par
$\cos(a \pm b) = \cos a \cos b \mp \sin a \sin b$\par
$\tan(a \pm b) = \frac{\tan(a)\pm\tan(b)}{1 \mp \tan(a)\tan(b)}$\par
$\tan(a) \pm \tan(b) = \frac{\sin(a \pm b)}{\cos(a)\cos(b)}$\par
$\sin(a) + \sin(b) = 2\sin(\frac{a + b}{2})\cos(\frac{a - b}{2})$\par
$\sin(a) - \sin(b) = 2\cos(\frac{a + b}{2})\sin(\frac{a - b}{2})$\par
$\cos(a) + \cos(b) = 2\cos(\frac{a + b}{2})\cos(\frac{a - b}{2})$\par
$\cos(a) - \cos(b) = -2\sin(\frac{a + b}{2})\sin(\frac{a - b}{2})$\par
$\sin(na) = n\cos^{n-1}a\sin a - \binom{n}{3}\cos^{n-3}a \sin^3a + \binom{n}{5}\cos^{n-5}a\sin^5a - \dots$\par
$\cos(na) = \cos^{n}a - \binom{n}{2}\cos^{n-2}a \sin^2a + \binom{n}{4}\cos^{n-4}a\sin^4a - \dots$\par


%\end{CJK*}
\end{document}